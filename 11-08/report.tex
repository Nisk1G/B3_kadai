%\documentstyle[epsf,twocolumn]{jarticle}       %LaTeX2.09仕様
%\documentclass[twocolumn]{jarticle}     %pLaTeX2e仕様
\documentclass{jarticle}     %pLaTeX2e仕様

%一枚組だったら[twocolumn]関係のとこ消す

\setlength{\topmargin}{-45pt}
%\setlength{\oddsidemargin}{0cm} 
\setlength{\oddsidemargin}{-7.5mm}
%\setlength{\evensidemargin}{0cm} 
\setlength{\textheight}{24.1cm}
%setlength{\textheight}{25cm} 
\setlength{\textwidth}{17.4cm}
%\setlength{\textwidth}{172mm} 
\setlength{\columnsep}{11mm}

\kanjiskip=.07zw plus.5pt minus.5pt

\usepackage{graphicx}
\usepackage[dvipdfmx]{color}
\usepackage{subcaption}
\usepackage{enumerate}
\usepackage{comment}
\usepackage{url}
\usepackage{multirow}
\usepackage{diagbox}


\begin{document}
  \noindent
  \onecolumn
  \hspace{1em}
  \today
  \hfill
  \ \  B3 西村昭賢 

  \vspace{2mm}
  \hrule
  \begin{center}
  {\Large \bf B3実験 11月8日分課題}
  \end{center}
  \hrule
  \vspace{3mm}

\section{課題1}
\subsection{実験内容}
gensim の Word2Vec ライブラリを使い, 与えられたテキストを学習した.

\subsection{前処理 と モデルのパラメータ}
与えられたテキストにおいて .  !  ?  ' のみ残し,その他の半角記号を消去した.その後  . ! ? の3つの記号に関しては前後に空白を挟むように調整し,文章中の全単語を小文字にした.
そして, Natural Language Toolkit (nltk) を用いて分かち書きした. その後  . ! ? の記号を文末と定義して 1 文ごとに分けて Word2Vec のモデルの学習を行えるようにした.
また, 表 に今回作成したWord2Vecモデルのパラメータを示す

\begin{table}[h]
  \caption{Word2Vecにおけるパラメータ}
  \label{Word2Vec}
  \centering
  \begin{tabular}{c|ccc}
    \hline
    パラメータ     & 分散表現の次元  &  学習時に利用される文脈の広さ    & 分散表現を獲得する単語の最小頻度 \\
    値 &  500 & 5 & 1 \\
    \hline
  \end{tabular}
\end{table}



\subsection{実験結果}
適当な名詞,動詞,形容詞を 1 つ選択しその単語と近い類似度の単語を5個調べた.
今回は名詞として 'alice' , 動詞として 'think' , 形容詞として 'good' を選択した.表 \ref{alice} にそれぞれの結果を示す.

\begin{table}[h]
  \caption{'alice'に最も類似度が高い上位 5 単語}
  \label{alice}
  \centering
  \begin{tabular}{l|ccccc}
    \hline
    単語       & 'trouble'  & 'wish'     & 'what'     & 'little'   & 'then'     \\
    類似度 & 0.130898 & 0.124550 & 0.104635 & 0.094204 & 0.092406 \\
    \hline
  \end{tabular}
\end{table}




\section{???}


\section{???}


%index.bibはtexファイルと同階層に置く
%ちゃんと\citeしないと表示されない(1敗)
\bibliography{index.bib}
\bibliographystyle{junsrt}

\end{document}