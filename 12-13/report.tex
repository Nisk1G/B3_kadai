%\documentstyle[epsf,twocolumn]{jarticle}       %LaTeX2.09仕様
%\documentclass[twocolumn]{jarticle}     %pLaTeX2e仕様
\documentclass{jarticle}     %pLaTeX2e仕様

%一枚組だったら[twocolumn]関係のとこ消す

\setlength{\topmargin}{-45pt}
%\setlength{\oddsidemargin}{0cm} 
\setlength{\oddsidemargin}{-7.5mm}
%\setlength{\evensidemargin}{0cm} 
\setlength{\textheight}{24.1cm}
%setlength{\textheight}{25cm} 
\setlength{\textwidth}{17.4cm}
%\setlength{\textwidth}{172mm} 
\setlength{\columnsep}{11mm}

\kanjiskip=.07zw plus.5pt minus.5pt

\usepackage{graphicx}
\usepackage[dvipdfmx]{color}
\usepackage{subcaption}
\usepackage{enumerate}
\usepackage{comment}
\usepackage{url}
\usepackage{multirow}
\usepackage{diagbox}


\begin{document}

  \noindent
  \hspace{1em}

  \today
  \hfill
  \ \  B3 西村昭賢 

  \vspace{2mm}
  \hrule
  \begin{center}
  {\Large \bf B3実験 12/13 進捗}
  \end{center}
  \hrule
  \vspace{3mm}


\section{時間内の目標}
\begin{quote}
  \begin{itemize}
   \item 研究発表会の準備(スライド)
   \item 研究発表会の準備(資料)
  \end{itemize}
 \end{quote}

\section{達成できたこと}
午前中のゼミで発表練習をし,森先生や先輩方から指摘を頂いた部分を修正・加筆した.
発表資料に関しても以前のB3 実験で要素技術まで書いていることからスライドを参考に発表の流れに沿って完成する目処がたった.



\section{来週作業予定}
環境改変後の強化学習によるエージェント作成の実験,構築環境のゲームバランス調整に取り組めていたら良い.



%index.bibはtexファイルと同階層に置く
%ちゃんと\citeしないと表示されない(1敗)
\bibliography{index.bib}
\bibliographystyle{junsrt}

\end{document}